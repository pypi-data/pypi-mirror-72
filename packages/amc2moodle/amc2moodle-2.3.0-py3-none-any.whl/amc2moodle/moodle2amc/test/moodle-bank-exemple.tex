\documentclass[a4paper]{article}
% -------------------------::== package ==::---------------------------
\usepackage[utf8]{inputenc}
\usepackage[T1]{fontenc}
\usepackage{alltt}
\usepackage{multicol}
\usepackage{amsmath,amssymb}
\usepackage{color}
\usepackage{graphicx}
% Mandatory for conversion
\usepackage[francais,bloc,completemulti]{automultiplechoice}
\usepackage{tikz}
\usepackage{hyperref}
\usepackage{ulem} % strike text

% -----------------------::== newcommand ==::--------------------------
\newcommand{\feedback}[1]{}
\begin{document}

% -----------------------------------------------------------------------------
\element{Défaut pour BN1-moodle2amc}{
  \begin{question}{Essai}\label{q:Essai}   
    Explain in few words the aim of this course.
% It is possible to add more granularity with partially correct answer using \wrongchoice[P]{p}\scoring{0.5}
\AMCOpen{lines=3}{    \correctchoice[OK]{OK}    \wrongchoice[F]{F}}
  \end{question}
}

% -----------------------------------------------------------------------------
\element{Défaut pour BN1-moodle2amc}{
  \begin{question}{html layout}\label{q:html layout}   
    a link \href{https://github.com/nennigb/amc2moodle}{here } and an image \includegraphics[]{./Figures/4.png} and an equation \( \int_{2\pi} x^2 \mathrm{d} x \)
\begin{center}
    centered text
\end{center}
flush left text
\begin{flushright}
    flush right text
\end{flushright}
In moodle editor, there is also \textsubscript{exponent} and \textsuperscript{indice} and \sout{that}and svg file \includegraphics[width=100px]{./Figures/dessin.png} 
  \begin{choices}
	    \correctchoice{This is the good \underline{underlined} answer.}    \wrongchoice{This is one \textit{italic} wrong answer.}    \wrongchoice{This a wrong \textbf{bold} answer.}
  \end{choices}

  \end{question}
}

% -----------------------------------------------------------------------------
\element{Défaut pour BN1-moodle2amc}{
  \begin{question}{table}\label{q:table}   
    Test html table conversion to tex        
\begin{center}
	
  \begin{tabular}{cccc}
  \hline
   & weight & width & length\\   
  \hline
sys1 & 1 kg & 0.35 m & 1 m\\   
sys2 & 2 kg & - & 1.5 m\\   
  \hline
	
  \end{tabular}\\
 table legend
\end{center}
 
  \begin{choices}
	    \wrongchoice{wrong answer}    \correctchoice{the good answer is obviously a weird table        
\begin{center}
	
  \begin{tabular}{cc}
  \hline
  stuff1 & stuff2\\   
  \hline
\(x^2\) & bold \textbf{text}\\   
  \hline
	
  \end{tabular}\\
 
\end{center}
 }    \wrongchoice{an other table, more simple         
\begin{center}
	
  \begin{tabular}{ccc}
  \hline
    Firstname & Lastname & Age\\   
  Jill & Smith & 50\\   
  Eve & Jackson & 94\\   
  \hline
	
  \end{tabular}\\
 
\end{center}
 }
  \end{choices}

  \end{question}
}

% ============================================================================
\exemplaire{1}{    	% nombre de sujet différent

% Replace with your Header
\vspace*{.5cm}
\begin{minipage}{.4\linewidth}
    \centering\large\bf Test
\end{minipage}
\champnom{\fbox{
    \begin{minipage}{.5\linewidth}
Nom et prénom :

\vspace*{.5cm}\dotfill
\vspace*{1mm}
    \end{minipage}
}}

\begin{center}
  \Large{\textsc{An AMC quiz generated from moodle XML questions export}}\\
  \normalsize
\end{center}

% mélange et catégorie (groupe dans ACM)
\cleargroup{allquestions}
\copygroup{Défaut pour BN1-moodle2amc}{allquestions}
\melangegroupe{allquestions}
\restituegroupe{allquestions}
}
\end{document}