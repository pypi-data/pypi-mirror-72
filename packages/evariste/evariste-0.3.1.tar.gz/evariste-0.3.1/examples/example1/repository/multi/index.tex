\documentclass[12pt]{article}

\usepackage{makeidx}
\makeindex

\usepackage[utf8x]{inputenc}
\usepackage[francais]{babel}

\begin{document}

À la mort de M. le duc de Bourgogne\index{Bourgogne@Bourgogne, Duc de},
lorsqu'il fut question d'aller jeter de l'eau bénite, le feu Roi\index{Louis
\textsc{xiv}} décida que si les princes lorrains s'y presenteroient, qu'eux
ni les ducs n'en jetteroient ; mais que si MM. de Rohan\index{Rohan} et de
Bouillon\index{Bouillon} y étoient, les ducs jetteroient de l'eau bénite
avant eux : ce qui arriva effectivement ; mais MM. de Rohan\index{Rohan} et
de Bouillon\index{Bouillon}, voyant les ducs passer avant eux, s'en allèrent.
Ce qui avoit été décidé en faveur de MM. les ducs fut écrit sur le registre
de M. de Dreux\index{Dreux} ; mais deux ans après, les représentations de Mme
de Maintenon\index{Maintenon} déterminèrent le Roi\index{Louis \textsc{xiv}}
à faire un changement et à ordonner à M. de Dreux\index{Dreux} que cet
article seroit rayé sur le registre. Il fut mis en marge que le
Roi\index{Louis \textsc{xiv}} n'avoit jamais voulu décider entre les ducs et
MM. de Rohan\index{Rohan} et de Bouillon\index{Bouillon}.

\printindex

\end{document}
